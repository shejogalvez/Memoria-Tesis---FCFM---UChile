% se puede agregar la opción [english] para 
%  memorias o tesis en inglés (borrando el archivo .aux)
\documentclass{umemoria} 

\depto{Departamento de Ciencias de la computación}
\author{Sergio Bastián Gálvez Leiva}
\title{Diseño e Implementación de Base de Datos de Grafos Aplicada a Museo de la Universidad de Chile}

% incluir ambos comandos para una doble titulación
%  o quitar el comando que no aplica
\memoria{Ingeniero Civil en Computación}
%\tesis{Doctor en ???} % incluir solo este comando para doctorados

% puede haber varios profesores guía seperados por coma;
% pero si es una memoria, solo puede haber un profesor guía
\guia{Sebastian Ferrada, Benjamín Bustos}

% puede haber varios profesores co-guía seperados por coma;
% pero si es una memoria, el profesor co-guía será el primer
% integrante de la comisión
%\coguia{Nombre Completo Co-Guía} % incluir en caso de co-guía de *tesis*

%\cotutela{Nombre Institución} % incluir en caso de cotutela

%\comision{Nombre Completo Uno,Nombre Completo Dos,Nombre Completo Tres}

%\auspicio{Nombre Institución} % incluir en caso de recibir financiamiento

% tiene que ser el año en que se da el examen de título/grado (defensa)
%\anho{2021} % incluir solo para reemplazar el año actual

\usepackage{lipsum}

\begin{document}

\frontmatter
\maketitle

\begin{resumen}
\lipsum[1-2]
\end{resumen}

% opcional: incluir para tesis en inglés;
%  en este caso hay que tener el resumen y abstract
%   en ambos idiomas
%\begin{abstract}
%\lipsum[1-4]
%\end{abstract}


\tableofcontents


\mainmatter

\chapter{Introducción}

\section{Contexto}

El trabajo de la presente memoria se va a realizar en el Museo de Arte Popular Americano Tomás Lago (MAPA), de la Facultad de Artes de la Universidad de Chile. Se va a trabajar en conjunto con la Directora
del Museo Constanza Urrutia y con una de las encargadas de ``Investigación y Colecciones'' Paula Cabrera.

Para contextualizar sobre la misión de MAPA y su historia, antes de su inauguración, en la década de 1930 se propiciaron las primeras iniciativas que buscaron rescatar, promover y difundir el desarrollo de las artes populares. Esto conlleva a que en 1943, en el contexto de los 100 años de la Universidad de Chile, se realizara la “Exposición Americana de Artes Populares” que contó con muchas piezas de diversos lugares de Latinoamérica. Luego el Consejo Universitario crea MAPA y se inaugura oficialmente el 20 de diciembre de 1944, donde hasta la fecha continúa su misión de rescatar, investigar y difundir el patrimonio que custodia.

Adicionalmente, la solución que se va a proponer en esta memoria también está enfocada a facilitar la conexión con distintos endpoints, sirviendo así, tanto para la información que pueda estar disponible en sus plataformas, como para facilitar la exportación de datos para entidades externas.

\section{Descripción del Problema y Situación Actual}

El problema que se va a abordar en esta memoria consiste principalmente en la poca utilización o nula existencia de una base de datos centralizada donde se puedan manejar tanto, los datos que manejan internamente dentro de la organización del museo, como los datos que comparten a terceros, particularmente a SURDOC \cite{surdoc}, que es una herramienta informática gubernamental dedicado a registrar y documentar información de los objetos que se encuentran en los museos de Chile.

La necesidad inmediata de realizar esta memoria es facilitar el trabajo de los trabajadores, colaboradores, investigadores, etc. que trabajan con MAPA y también mejorar la capacidad de trabajo para ayudar en la misión de MAPA.

Respecto a la situación actual de MAPA, tienen una colección de alrededor de 6700 piezas, de las cuales deben anotar una serie de datos, como por ejemplo, nombre del objeto, cultura originaria, dimensiones, material, imagen del objeto, etc. (en la página de SURDOC se puede ver en detalle los datos que deben exportar para cada objeto), Estos datos están registrados en planillas Excel, donde distintos equipos dentro de MAPA manejan distintos datos sobre los objetos. Esto dificulta enormemente el acceso a la información de los objetos y el poder exportar estos datos a SURDOC. De aquí nace la necesidad de proponer un sistema de administración de datos que le permite al museo MAPA manejar sus datos y poder exportarlos en forma simple y eficiente.

La exportación de datos a SURDOC se hace ingresando en un formulario digital, que se completa de manera independiente para cada una de las piezas, los datos obligatorios que se solicitan (lo que incluye al menos un registro fotográfico). Una vez se completa y envía el formulario el sistema genera un número de inventario para la pieza en el sistema. SURDOC autoriza a uno de los trabajadores de MAPA para hacer el ingreso de estos datos y le entrega un usuario/perfil para completar esta formulario de varias páginas.

\section{Objetivos}\label{chap:obj}

\subsection*{Objetivo General}\label{sec:obj-g}

El objetivo principal de este proyecto es crear un grafo de conocimiento que pueda responder a la necesidad de todas las personas que utilizan los datos asociados a MAPA, que logre una usabilidad satisfactoria para la contraparte de MAPA y que agregue funcionalidades para reducir el trabajo manual.

\subsection*{Objetivos Específicos}\label{sec:obj-e}


\begin{enumerate}
\item Diseñar un grafo de conocimiento que capture toda la información disponible en MAPA respecto a sus piezas y cumpla con los requisitos de los encargados de MAPA % definir bien el modelo y usar el motor apropiado

%\item disminuir el tiempo que ocupan los usuarios tanto para poder ingresar datos como para recuperar datos 

\item Implementar grafo de conocimiento e ingresar datos existentes de MAPA a nueva BDD.

\item Implementar exportación automática de datos en formatos compatibles con otros sistemas, particularmente para poder hacer envío de estos datos a SURDOC.

\item Implementar una interfaz básica que cumpla con requisitos mínimos para interactuar con la BDD
\end{enumerate}
\chapter{Estado del Arte}

\section{Gestores de Bases de Datos}

Entre los distintos modelos de bases de datos, las bases de datos relacionales (RDB) han sido el estándar de facto en la industria durante décadas y siguen siendo ampliamente utilizadas en una variedad de aplicaciones empresariales. Aunque han surgido alternativas, las RDB siguen siendo populares debido a su amplia adopción.

Pero aunque los motores de base de datos, particularmente los más usados, sean principalmente relacionales (RDBMS) \cite{DB_ranking}, también son multi-modelo, es decir, ofrecen funcionalidades para administrar datos como otro modelo de datos u otros tipos de bases de datos, entre estos las bases de datos de grafos.

Las bases de datos de grafos (GDB) han ganado popularidad en los últimos años \cite{Goasduff_2021}, especialmente en aplicaciones donde la estructura y la interconexión de los datos son fundamentales. Aunque aún no han alcanzado la misma adopción masiva que las RDB, su uso está en constante crecimiento, especialmente en áreas como redes sociales, análisis de redes y recomendaciones \cite{aryono2016modelling, DBLP:journals/tgis/ParkC23, 9216015}. 

Las GDB tienen más flexibilidad en las relaciones entre datos, los nodos pueden cambiar de \textit{tag} (equivalente a tabla en RDB), pueden ser eliminados, se pueden agregar atributos a un nodo y no va a afectar otras relaciones, las entidades no tienen un esquema definido, a diferencia de las RDB que se caracterizan por tener un esquema rígido. Permite hacer operaciones de grafos sobre la data (búsqueda de caminos, distancia entre nodos), tiene aplicaciones en datos que se puedan modelar como una red. Es este punto el principal motivo que justifica la decisión de usar un modelo de grafos en vez de algo más convencional como lo es un modelo relacional.

\section{Bases de datos de Grafos (GDB)}

\section{Gestores de GDB}
\chapter{Solución}
\section{Caracterización de datos}\label{sec:obj-g}

copiar lo realizado en propuesta, usar grafo actualizado

\section{Elección de Gestor de Base de Datos}\label{sec:obj-e}

justificar elección de graph DB

\subsection{MilleniumDB}

\begin{itemize}
    \item descripción
    \item pros y cons
    \item concluir que se buscó alternativa
\end{itemize}

\subsection{Neo4j}

\begin{itemize}
    \item descripción
    \item pros y cons
    \item es suficiente
\end{itemize}

\section{Estructura de la Solución}

\subsection{Docker}
\begin{itemize}
    \item por que se usó docker
\end{itemize}

\section{Backend}

\subsection{FastAPI}
\subsection{manejo de consultas}
\subsection{manejo de usuarios}
\subsection{log de cambios}
\subsection{exportación a SURDOC}

\section{Frontend}
\subsection{piezas}
\subsection{login}

\section{Evaluación y Testeo}
\chapter{Conclusión}

\lipsum[130-132]
\begin{figure}
	\centering
	\includegraphics[scale=.2]{imagenes/fcfm.pdf}
	\caption{Logo de la Facultad}
	\label{logofcfm}
\end{figure}

\lipsum[133-134]



% ver https://www.overleaf.com/learn/latex/Glossaries
% \input{glosario.tex} % opcional

\nocite{*}
\bibliographystyle{plain}
\bibliography{bibliografia}

% opcional ...
\begin{appendices}
\input{anexoA.tex}
\end{appendices}
\end{document}